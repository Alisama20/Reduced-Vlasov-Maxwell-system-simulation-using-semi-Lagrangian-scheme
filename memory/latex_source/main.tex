\documentclass[11pt,a4paper,twoside,pdf]{article}

% Codificación y tipografía
\usepackage[T1]{fontenc}
\usepackage[utf8]{inputenc}
\usepackage{mathpazo}

% Matemáticas
\usepackage{amsmath, amssymb, amsfonts}

% Figuras
\usepackage{graphicx}
\usepackage{float}
\usepackage{subcaption}

% Tablas y utilidades
\usepackage{multirow}
\usepackage{rotating}
\usepackage{array}
\usepackage{booktabs}
\usepackage{xspace}

% Símbolos adicionales
\usepackage{eurosym}
\usepackage{bbding}
\usepackage{pifont}

% Otros
\usepackage{soul}
\usepackage{color}
\usepackage{latexsym}
\usepackage{xtab}

% Hipervínculos
\usepackage[colorlinks=true,urlcolor=blue,linkcolor=blue,citecolor=blue]{hyperref}

% Numeración de ecuaciones
\numberwithin{equation}{section}

% Márgenes
\usepackage[top=2.88cm,bottom=2.97cm,left=2.95cm,right=2.95cm]{geometry}

% Idioma
\usepackage[spanish,es-nodecimaldot,es-tabla,es-lcroman,es-nosectiondot,es-noindentfirst]{babel}

% Cabeceras
\usepackage{fancyhdr}

% Teoremas
\newtheorem{theorem}{Teorema}[section]
\newtheorem{corollary}{Corolario}[theorem]
\newtheorem{lemma}[theorem]{Lema}

% Comandos propios
\newcommand{\dis}{\displaystyle}

\linespread{1.05}

\begin{document}

% Portada %%%%%%%%%%%%%%%%%%%%%%%%%%%%%%%%%%%%%%%%%%%%%%%%%%%%%%%%%%%%%%%%%%%%%%

\pagestyle{empty}


\noindent
\begin{tabular}{r}
\includegraphics[width=8.8cm]{escudoUGRmonocromo.png} \\[-1.8ex]
\hspace{31mm}\vspace{-8mm}
\begin{tabular}{c}
\hline\\[-1ex]\hskip-2mm
{\bf Facultad de Ciencias}\hspace{18mm}
\end{tabular}
\end{tabular}

\large
\vspace{30mm}
\hspace{25mm}
\begin{tabular}{l}
MÁSTER EN FÍSICA Y MATEMÁTICAS
\end{tabular}

\vspace{45mm}
\hspace{25mm}
\begin{tabular}{l}
EDPs DE TRANSPORTE EN TEORÍA CINÉTICA \\
Y MECÁNICA DE FLUIDOS
\\[1.5ex]
\LARGE\bf VLASOV-MAXWELL
\end{tabular}
%
\vfill
\hspace{25mm}
\begin{tabular}{l}
Presentado por:
\\
{\bf A. S. Amari }\\

\textcolor{blue}{\texttt{alisalemstd@gmail.com}, \texttt{alisalemstd@correo.ugr.es}}

\\[3ex]
Curso Académico 2025/2026

\end{tabular}
%

\newpage
%


% Indice %%%%%%%%%%%%%%%%%%%%%%%%%%%%%%%%%%%%%%%%%%%%%%%%%%%%%%%%%%%%%%%%%%%%%%%
%\newpage

\tableofcontents

% Texto %%%%%%%%%%%%%%%%%%%%%%%%%%%%%%%%%%%%%%%%%%%%%%%%%%%%%%%%%%%%%%%%%%%%%%%%
%\newpage

\pagestyle{fancy}
\fancyhead[RO,LE]{\leftmark}
\fancyhead[LO,RE]{\thepage}
\fancyfoot{}








\newpage

\section{El sistema de Valsov--Maxwell}

Los plasmas constituyen un estado de la materia que se caracteriza por la
presencia de partículas cargadas que se mueven a elevadas velocidades.
A diferencia de los fluidos neutros, la dinámica de un plasma está gobernada por la
evolución conjunta de las partículas y de los campos que ellas mismas generan, lo que
da lugar a un sistema fuertemente acoplado. Debido a la elevada cantidad de partículas del sistema (del orden del número de Avogadro $N_A \sim 10^{23}$) la descripción adecuada es de naturaleza cinética y se basa en una
función de distribución de probabilidad $f(t,x,p)$ que describe la densidad de partículas en el espacio
de fases. Además, también es necesario introducir una formulación adicional para las interacciones electromagnéticas entre las partículas.





%----------------------------------------------------





\subsection{Dinámica microscópica y ecuación de Vlasov}

La descripción microscópica de un plasma parte del hecho fundamental de que está
formado por un número extremadamente grande de partículas cargadas (electrones e
iones) que interactúan entre sí mediante fuerzas de largo alcance. Desde un punto
de vista estrictamente newtoniano, el estado de cada partícula $i$ queda determinado
por su posición $x_i(t)$ y su velocidad $v_i(t)$, cuya evolución viene dada por:
\begin{equation}
\begin{cases}
x'_i(t) = v_i(t), \\[0.3em]
m\, v'_i(t) = F\big(t,x_i(t),v_i(t)\big),
\end{cases}
\label{eq:newton_micro}
\end{equation}
donde $F(t,x,v)$ representa el campo de fuerzas total que actúa sobre la partícula.
Este campo puede depender explícitamente del tiempo, de la posición y de la
velocidad, lo que permite incluir tanto fuerzas externas como interacciones
electromagnéticas generadas por el propio plasma.

Sin embargo, cuando el número de partículas $N$ es del orden del número de
Avogadro, el sistema \eqref{eq:newton_micro} resulta completamente inabordable.
Incluso si fuera posible resolver las ecuaciones de Newton para cada partícula, la
información obtenida sería inútil desde un punto de vista físico: lo que interesa no
es la trayectoria individual de cada electrón, sino el comportamiento colectivo del
conjunto. Esta observación motiva la transición hacia una descripción estadística.

\medskip

En lugar de seguir cada partícula de manera individual, se introduce la función de
distribución


\[
f(t,x,v),
\]


que describe la densidad de partículas en el espacio de fases. Para cualquier región
$A\subset\mathbb{R}^3_x$ y $B\subset\mathbb{R}^3_v$, la cantidad


\[
\int_A\int_B f(t,x,v)\,dx\,dv
\]


representa el número de partículas que en el instante $t$ se encuentran en $A$ con
velocidades en $B$. Esta función constituye el objeto central de la teoría cinética \cite{romeroedp}.

\medskip


Para deducir la ecuación que gobierna la evolución de $f$, se considera primero la distribución discreta exacta

\begin{equation}
f_{ex}(t,x,v)=\sum_{i=1}^N \delta(x-x_i(t))\,\delta(v-v_i(t)),
\label{eq:f_ex}
\end{equation}

(donde $\delta(x)$ es la distribución delta de Dirac) que representa el estado microscópico del sistema. Para cualquier función test
$\phi(x,v)$ suave y de soporte compacto, se tiene


\[
\int_{\mathbb{R}^3}\int_{\mathbb{R}^3} \phi(x,v)\,f_{ex}(t,x,v)\,dx\,dv
= \sum_{i=1}^N \phi(x_i(t),v_i(t)).
\]



Derivando respecto al tiempo y aplicando la regla de la cadena junto con
\eqref{eq:newton_micro}, se obtiene


\[
\frac{d}{dt}\int \phi\,f_{ex}
= \sum_{i=1}^N \left[
v_i(t)\cdot\nabla_x\phi(x_i(t),v_i(t))
+ \frac{1}{m}F(t,x_i(t),v_i(t))\cdot\nabla_v\phi(x_i(t),v_i(t))
\right].
\]



Usando nuevamente la propiedad de la delta de Dirac, esta expresión puede
reescribirse como


\[
\int \phi\,\partial_t f_{ex}
= \int \left[
v\cdot\nabla_x\phi(x,v)
+ \frac{1}{m}F(t,x,v)\cdot\nabla_v\phi(x,v)
\right] f_{ex}(t,x,v)\,dx\,dv.
\]



\medskip

Para obtener la ecuación diferencial que satisface $f_{ex}$, se realiza una
integración por partes en el término de la derecha. Como $\phi$ tiene soporte
compacto, los términos de frontera se anulan, y se obtiene


\[
\int \left[
\partial_t f_{ex}
+ \operatorname{div}_x(v f_{ex})
+ \operatorname{div}_v\!\left(\frac{F(t,x,v)}{m} f_{ex}\right)
\right]\phi\,dx\,dv = 0.
\]



Dado que esta igualdad debe verificarse para toda función test $\phi$, $f_{ex}$ satisface en sentido débil la ecuación


\[
\partial_t f_{ex}
+ \operatorname{div}_x(v f_{ex})
+ \operatorname{div}_v\!\left(\frac{F(t,x,v)}{m} f_{ex}\right)=0.
\]



Finalmente, al pasar al límite continuo $f_{ex}\to f$ cuando $N\to\infty$, se obtiene
la ecuación de Vlasov en su forma conservativa general:
\begin{equation}
\partial_t f + v\cdot\nabla_x f
+ \operatorname{div}_v\!\left(\frac{F(t,x,v)}{m}f\right)=0.
\label{eq:vlasov_general}
\end{equation}

\medskip


La ecuación \eqref{eq:vlasov_general} expresa que la densidad de partículas se
transporta a lo largo de las trayectorias determinadas por el campo vectorial
$(v,F/m)$ en el espacio de fases. No contiene términos difusivos ni disipativos,
por lo que el modelo describe plasmas \emph{sin colisiones} \footnote{A este respecto, se puede introducir un término a la derecha de la equación del tiempo $\left( \frac{\partial f}{\partial t} \right)_c$ que expresa la tasa de cambio de $f$ debido a colisiones \cite{chen1984introduction}.}. La dinámica es
puramente determinista y conserva la densidad a lo largo del flujo, lo que convierte
a la ecuación de Vlasov en una ley de conservación en el espacio de fases.



%------------------------------------------------------

\subsection{Acoplamiento electromagnético: ecuaciones de Maxwell}

La ecuación de Vlasov describe la evolución de la función de distribución bajo la
acción de un campo de fuerzas general $F(t,x,v)$. En el caso de plasmas, este campo
no es externo, sino que está generado por las propias partículas del
sistema. Esto introduce un acoplamiento fundamental: las partículas se mueven bajo
la acción de los campos electromagnéticos y dichos campos dependen a su vez de
la distribución de partículas.

Para partículas cargadas, el campo de fuerzas viene dado por la fuerza de Lorentz,
que en términos del campo eléctrico $E(t,x)$ y del campo magnético $B(t,x)$ se
escribe como \cite{martin2021campo}


\[
F(t,x,v) = q\big(E(t,x) + v \times B(t,x)\big),
\]


donde $q$ es la carga de la partícula. 



Los campos $E$ y $B$ no son arbitrarios: su evolución está gobernada por las
ecuaciones de Maxwell, que en su forma diferencial son
\begin{align}
\nabla \cdot E &= \frac{\rho}{\varepsilon_0}, \label{eq:gauss_e}\\
\nabla \cdot B &= 0, \label{eq:gauss_b}\\
\nabla \times E &= -\frac{\partial B}{\partial t}, \label{eq:faraday}\\
\nabla \times B &= \mu_0 j + \mu_0 \varepsilon_0 \frac{\partial E}{\partial t}. \label{eq:ampere}
\end{align}
Las ecuaciones \eqref{eq:gauss_e} y \eqref{eq:gauss_b} son ecuaciones de tipo
elíptico que imponen restricciones instantáneas sobre los campos, mientras que
\eqref{eq:faraday} y \eqref{eq:ampere} son ecuaciones evolutivas que describen la
propagación de las ondas electromagnéticas.


En un plasma descrito por una función de distribución $f(t,x,v)$, las fuentes de las
ecuaciones de Maxwell —la densidad de carga $\rho$ y la densidad de corriente $j$—
se obtienen como momentos de $f$
\begin{equation}
\rho(t,x) = q \int_{\mathbb{R}^3} f(t,x,v)\,dv, 
\qquad
j(t,x) = q \int_{\mathbb{R}^3} v\, f(t,x,v)\,dv.
\label{eq:rho_J}
\end{equation}
\\


\newpage
El acoplamiento entre la ecuación de Vlasov y las ecuaciones de Maxwell da lugar al
sistema de Vlasov–Maxwell: la función de distribución evoluciona bajo la acción de los campos
electromagnéticos, mientras que estos campos se determinan a partir de las cargas y
corrientes inducidas por la propia distribución. Desde un punto de vista físico, el sistema de Vlasov–Maxwell combina dos niveles de
descripción: la dinámica microscópica de las partículas, gobernada por la ecuación de Vlasov y la dinámica macroscópica de los campos, gobernada por las ecuaciones de Maxwell.\newline

La interacción entre ambos niveles es lo que permite capturar la complejidad de los
plasmas sin colisiones, donde los efectos colectivos dominan sobre las colisiones
partícula–partícula. Este marco constituye el punto de partida para la derivación de
modelos reducidos, como el sistema unidimensional que se estudia en las secciones
posteriores.


%----------------------------------------------------------------


\section{Dinámica relativista}





En aplicaciones de interacción láser–plasma, las partículas pueden alcanzar
velocidades comparables a la de la luz. En estas condiciones, la relación clásica entre
velocidad y momento deja de ser válida, y es necesario recurrir a la formulación
relativista de la dinámica \cite{janssen2022gravitacion}. El uso del momento lineal $p$ como variable fundamental
resulta especialmente conveniente, ya que permite expresar la velocidad mediante la
relación relativista


\[
v(p)=\frac{p}{m\gamma(p)}, \qquad
\gamma(p)=\sqrt{1+\frac{|p|^2}{m^2c^2}}.
\]




Además, la descripción electromagnética se expresa de manera natural en términos
de los potenciales electromagnéticos: un potencial escalar $\Phi$ y un potencial vector
$A$, a partir de los cuales se obtienen los campos eléctrico y magnético. \newline

El objetivo de esta sección es presentar los elementos fundamentales de la dinámica
relativista de plasmas que motivan el modelo reducido de Vlasov–Maxwell estudiado
en este trabajo. Se introduce la relación entre momento y velocidad
en el marco relativista, para después describir el papel de los potenciales
electromagnéticos y, finalmente, justificar la reducción unidimensional que conduce a
los modelos no relativista (NR), cuasi–relativista (QR) y plenamente relativista (FR).


%----------------------------------------------------------------


\subsection{Potenciales electromagnéticos}

La formulación clásica de las ecuaciones de Maxwell utiliza los campos eléctrico
$E(t,x)$ y magnético $B(t,x)$ como variables fundamentales. Sin embargo, en tratamientos relativistas resulta más conveniente trabajar con los \emph{potenciales
electromagnéticos}: un potencial escalar $\Phi(t,x)$ y un potencial vector $A(t,x)$.
Estos potenciales permiten expresar los campos mediante las relaciones \cite{martin2021campo}
\begin{equation}
B = \nabla \times A, \qquad
E = -\nabla \Phi - \partial_t A.
\label{eq:potenciales}
\end{equation}


\\


El uso de potenciales electromagnéticos presenta varias ventajas en la derivación de
modelos reducidos de Vlasov–Maxwell permite incorporar de forma natural la estructura geométrica del campo
magnético, especialmente en configuraciones donde el campo es transversal a
la dirección de propagación. Además, facilita la reducción dimensional, ya que en una dimensión espacial el campo magnético puede expresarse mediante una única componente del potencial vector, por lo que se simplifica la escritura del término de fuerza en la ecuación de Vlasov, al expresar $E$ y $B$ directamente en términos de $A$ y $\Phi$ \cite{carrillo2006global}.





%--------------------------------------------------------------



\subsection{Reducción unidimensional y modelos relativistas}

El sistema de Vlasov–Maxwell en tres dimensiones describe con exactitud la
dinámica de un plasma collisionless. Sin embargo, su complejidad matemática y
computacional hace que, en muchas aplicaciones, resulte necesario derivar modelos
efectivos que capturen los fenómenos esenciales con un coste más reducido. Este es
el caso de la interacción láser–plasma, donde la geometría del problema y las
escalas temporales permiten realizar simplificaciones \cite{ghizzo1990vlasov}.\\



En sucesos de propagación de ondas láser en plasmas, la dinámica
dominante ocurre a lo largo de la dirección de propagación del haz, que se denota por $x$. Los fenómenos transversales —como la difusión lateral — son mucho más lentos y de menor amplitud. Esta separación de
escalas permite suponer que todas las magnitudes relevantes dependen únicamente
de la coordenada longitudinal:


\[
f(t,x,p) \equiv f(t,x,p_x,p_\perp), \qquad
E(t,x), \qquad B(t,x).
\]


Bajo esta hipótesis, el sistema tridimensional se reduce a un modelo en una dimensión
espacial, manteniendo la dependencia completa en el momento.



En una dimensión espacial, el campo magnético solo puede tener componentes
transversales. Por tanto, el potencial vector $A$ se reduce a una única componente
$A(t,x)$, que satisface una ecuación de onda acoplada con la densidad.
El campo eléctrico longitudinal se obtiene a partir de la ecuación de Poisson, mientras
que la componente transversal del campo eléctrico se expresa como $-\partial_t A$ \cite{martin2021campo}.\\


La relación entre velocidad y momento en el marco relativista puede simplificarse en función del régimen físico
considerado. Esto conduce a tres versiones del modelo \cite{janssen2022gravitacion}:

\begin{itemize}
\item \textbf{Modelo no relativista (NR):}  
Se asume que las velocidades son pequeñas comparadas con $c$, por lo que  


\[
\gamma = 1.
\]


El sistema resultante es una clase de soluciones exactas del modelo
Vlasov–Maxwell no relativista.

\item \textbf{Modelo cuasi–relativista (QR):}  
Se mantiene la relación relativista en el término de transporte,


\[
\gamma_1(p)=\sqrt{1+p^2},
\]


pero se toma $\gamma_2=1$ en los términos no lineales acoplados al potencial
vector. Este modelo captura los efectos relativistas esenciales sin introducir la
complejidad total del caso plenamente relativista.

\item \textbf{Modelo plenamente relativista (FR):}  
Se conserva la expresión completa del factor de Lorentz, que en el modelo
reducido depende tanto del momento como del potencial vector:


\[
\gamma(p,A)=\sqrt{1+p^2 + A^2}.
\]


Este modelo describe con precisión la dinámica relativista, pero presenta un
acoplamiento no lineal mucho más fuerte.
\end{itemize}


Los tres modelos comparten la misma estructura cinética, pero difieren en el grado
de relatividad incorporado. El modelo NR es matemáticamente más sencillo, pero
insuficiente en regímenes de alta energía. El modelo FR es físicamente más preciso,
pero su análisis teórico y su resolución numérica son considerablemente más
complejos. El modelo QR constituye un compromiso intermedio: conserva la
estructura relativista en el transporte, mantiene un acoplamiento manejable y es
ampliamente utilizado en simulaciones de interacción láser–plasma.






%-------------------------------------------------------------








\section{Resultados de existencia y unicidad}

Bajo las hipótesis físicas expuestas —unidimensionalidad,
monocineticidad transversal y uso del potencial vector— y tras un cambio de unidades físicas el sistema completo puede
escribirse en las variables $(t,x,p)$ como \cite{carrillo2006global}:

\begin{align}
\partial_t f
+ \frac{p}{\gamma_1}\,\partial_x f
- \left( E(t,x)
+ \frac{A(t,x)}{\gamma_2}\,\partial_x A(t,x) \right)\partial_p f &= 0, \label{eq:RM1}\\[0.4em]
\partial_{tt} A - \partial_{xx} A &= - \rho_{\gamma_2}(t,x)\,A(t,x), \label{eq:RM2}\\[0.4em]
\partial_t E &= j(t,x), \label{eq:RM3}\\[0.4em]
\partial_x E &= \rho_{\mathrm{ext}}(x) - \rho(t,x). \label{eq:RM4}
\end{align}

Las cantidades macroscópicas $\rho$, $\rho_{\gamma_2}$ y $j$ se obtienen como momentos
de la función de distribución:


\[
\{\rho,\,\rho_{\gamma_2},\,j\}(t,x)
= \int_{\mathbb{R}} \left\{ 1,\;\frac{1}{\gamma_2},\;\frac{p}{\gamma_1} \right\} f(t,x,p)\,dp.
\]






El término $\rho_{\mathrm{ext}}(x)$ representa la densidad de iones de fondo, que se
consideran inmóviles en la escala temporal del modelo. El sistema se completa con
las condiciones iniciales
\begin{equation}
f(0,x,p)=f_0(x,p), \quad
E(0,x)=E_0(x), \quad
A(0,x)=A_0(x), \quad
\partial_t A(0,x)=A_1(x).
\label{eq:IC}
\end{equation}

\subsection{Soluciones por características}

La ecuación de Vlasov \eqref{eq:RM1} es una ecuación de transporte cuyo flujo viene
determinado por las características \cite{romeroedp}


\[
(X(s),P(s)) = (X,P)(s;t,x,p),
\]


definidas por el sistema
\begin{equation}
\frac{dX}{ds} = \frac{P(s)}{\gamma_1(P,A)}, \qquad
\frac{dP}{ds} = -E(s,X(s)) - \frac{A(s,X(s))}{\gamma_2(P,A)}\,\partial_x A(s,X(s)),
\label{eq:char}
\end{equation}
con condiciones finales


\[
X(t;t,x,p)=x, \qquad P(t;t,x,p)=p.
\]



Una solución \emph{mild} del sistema reducido es aquella que satisface la ecuación de
Vlasov a lo largo de las características \eqref{eq:char} y que resuelve las ecuaciones
de Maxwell \eqref{eq:RM2}--\eqref{eq:RM4} en sentido débil. Con esta definición, a continuación se exponen los resultados obtenidos en \cite{bostan2007mild} sobre existencia y unicidad de soluciones de este tipo.



\begin{theorem}[Existencia y unicidad en los casos QR y FR]
Supóngase que $f_0\ge 0$ y que $(1+|p|^k)f_0\in L^1(\mathbb{R}^2)$, donde
$k=2$ en el caso QR y $k=1$ en el caso FR. Asúmase además que existe una función
$n_0:\mathbb{R}\to\mathbb{R}_+$, no decreciente en $\mathbb{R}_-$ y no creciente en
$\mathbb{R}_+$, tal que


\[
f_0(x,p)\le n_0(p), \qquad
\int_{\mathbb{R}} |p|^k n_0(p)\,dp <\infty, \qquad
\|n_0\|_{L^\infty}<\infty.
\]



Sea $\rho_{\mathrm{ext}}\in L^1(\mathbb{R})\cap L^\infty(\mathbb{R})$ no negativa, y supóngase
que las condiciones iniciales \eqref{eq:IC} satisfacen


\[
E_0' = \rho_{\mathrm{ext}} - \int_{\mathbb{R}} f_0\,dp, \qquad
A_0\in W^{2,\infty}(\mathbb{R}), \qquad
A_1\in W^{1,\infty}(\mathbb{R}).
\]



Entonces, el sistema \eqref{eq:RM1}--\eqref{eq:RM4} admite una única solución global
$(f,E,A)$ tal que, para todo $T>0$,


\[
f\ge 0, \qquad
(1+|p|^k)f \in L^\infty([0,T];L^1(\mathbb{R}^2)),
\]




\[
E\in W^{1,\infty}([0,T]\times\mathbb{R}), \qquad
A\in W^{2,\infty}([0,T]\times\mathbb{R}).
\]


Además, si $f_0\in W^{1,\infty}(\mathbb{R}^2)$, entonces $f\in W^{1,\infty}([0,T]\times\mathbb{R}^2)$.
\end{theorem}

Este resultado garantiza la existencia global de soluciones por características en los
casos QR y FR. En contraste, en el caso no relativista (NR) solo se dispone de
existencia local en el tiempo, debido a la falta de control sobre el crecimiento del
momento.\\

Para el análisis numérico —y en particular para el estudio de esquemas
semi--Lagrangianos— es necesario disponer de mayor regularidad sobre las soluciones.
Mediante argumentos similares a los empleados en el teorema anterior, se obtiene:

\begin{theorem}[Regularidad mejorada]
Bajo hipótesis adicionales de integrabilidad y acotación sobre $f_0$ y sus derivadas,
y suponiendo mayor regularidad sobre $\rho_{\mathrm{ext}}$, $A_0$ y $A_1$, la solución
$(f,E,A)$ del sistema reducido satisface


\[
(1+|p|^k)(f + |\nabla_{x,p} f|) \in L^\infty([0,T];L^1(\mathbb{R}^2)),
\]




\[
E\in W^{2,\infty}([0,T]\times\mathbb{R}), \qquad
A\in W^{3,\infty}([0,T]\times\mathbb{R}),
\]


y, si $f_0\in W^{2,\infty}$, entonces $f\in W^{2,\infty}$ para todo $T>0$.
\end{theorem}

Como consecuencia, si $f_0$ posee soporte compacto, dicho soporte permanece
uniformemente acotado para todo tiempo finito, lo que resulta esencial para el análisis
de esquemas numéricos en dominios no acotados.

\medskip

Estos resultados proporcionan el marco teórico necesario para el estudio de métodos
numéricos aplicados al modelo reducido de Vlasov--Maxwell. En particular, permiten
garantizar la existencia de características bien definidas y campos suficientemente
regulares para justificar la convergencia de esquemas semi--Lagrangianos, que serán
analizados en las secciones siguientes.



%----------------------------------------------------------------


\section{Método numérico: esquema semi-Langrangiano}

El objetivo de esta sección es presentar las ideas fundamentales del método
semi–Lagrangiano aplicado al modelo reducido de Vlasov–Maxwell.



La ecuación de Vlasov describe un transporte en el espacio de fases en el cuál la
función de distribución $f$ se conserva a lo largo de las características. El método
semi–Lagrangiano explota precisamente esta propiedad: para avanzar la solución en
el tiempo, se busca el punto característico que llega a $(x,p)$ en el instante $t^{n+1}$ y
se evalúa allí la solución en el instante anterior $t^n$.

Este enfoque evita restricciones de tipo Courant-Friedrichs-Lewy (como las que aparecen en métodos en diferencias finitas \cite{dymer2025stable}) y permite manejar desplazamientos
grandes en el espacio de fases, lo que lo convierte en una herramienta especialmente
adecuada para plasmas donde el rango de velocidades relativistas exige saltos de tiempo muy pequeños para capturar los elevados gradientes en el momento.



%---------------------------------------------------------------



\subsection{Aproximación de las características}

Para avanzar la solución de la ecuación de Vlasov en el tiempo, el método
semi–Lagrangiano se basa en la conservación de la función de distribución a lo largo
de las características 


\[
\frac{dX}{dt} = v(P), \qquad \frac{dP}{dt} = -F(t,X),
\]


donde, en el caso cuasi–relativista (QR),


\[
v(p) = \frac{p}{\gamma}, \qquad \gamma = \sqrt{1+p^2}, \qquad
F(t,x) = -(E(t,x) + A(t,x)\,\partial_x A(t,x)).
\]



El objetivo es aproximar el punto característico hacia atrás en el tiempo,


\[
(X,P)(t-\Delta t;t,x,p),
\]


para evaluar allí la solución en el instante anterior. Como resolver este sistema de
forma exacta es costoso, se utiliza un método de \emph{splitting} que separa la
advección en espacio y en momento.


El desplazamiento hacia atrás en el tiempo se aproxima mediante tres pasos:

\begin{enumerate}
\item \textbf{Media advección en espacio:}


\[
(x,p) \longmapsto \left(x - \frac{\Delta t}{2}v(p),\; p\right).
\]



\item \textbf{Advección completa en momento:}


\[
(x,p) \longmapsto \left(x,\; p - \Delta t\,F\!\left(t - \frac{\Delta t}{2},\, x\right)\right).
\]



\item \textbf{Media advección en espacio:}


\[
(x,p) \longmapsto \left(x - \frac{\Delta t}{2}v(p),\; p\right).
\]


\end{enumerate}

Combinando estos tres pasos, la aproximación del punto característico hacia atrás se
escribe explícitamente como:


\[
X(t-\Delta t;t,x,p) \approx x
- \frac{\Delta t}{2}v(p)
- \frac{\Delta t}{2}v\!\left(p - \Delta t\,F\!\left(t-\frac{\Delta t}{2},\, x - \frac{\Delta t}{2}v(p)\right)\right)
\]
\[
P(t-\Delta t;t,x,p) \approx p
- \Delta t\,F\!\left(t-\frac{\Delta t}{2},\, x - \frac{\Delta t}{2}v(p)\right).
\]



Para abreviar, se introduce la notación


\[
\tilde{F} = F\!\left(t-\frac{\Delta t}{2},\, x - \frac{\Delta t}{2}v(p)\right),
\]

de modo que las fórmulas anteriores pueden escribirse como


\[
\begin{aligned}
\tilde{p} &= p - \Delta t\,\tilde{F}, \\[0.3em]
\tilde{x} &= x - \frac{\Delta t}{2}v(p) - \frac{\Delta t}{2}v(\tilde{p}).
\end{aligned}
\]



Estas expresiones constituyen la base del método semi–Lagrangiano: una vez
calculado el punto $(\tilde{x},\tilde{p})$, la solución se actualiza mediante:


\[
f^{n+1}(x,p) = f^{n}(\tilde{x},\tilde{p})
\]


utilizando interpolación para evaluar $f^n$ en puntos no nodales.\\


A continuación se exponen los resultados obtenidos en \cite{bostan2009convergence} sobre el error en la aproximación por características y en la interpolación lineal.

\begin{lemma}[Error en la aproximación de características]
Sea $(X,P)$ la característica exacta y $(\tilde{x},\tilde{p})$ la aproximación obtenida por
splitting. Si el campo de fuerzas $F$ es suficientemente regular, entonces


\[
|\tilde{x}-X(t-\Delta t)| \le C\Delta t^3 + C\Delta t^2\|F-F_h\|_{L^\infty},
\]




\[
|\tilde{p}-P(t-\Delta t)| \le C\Delta t^3 + \Delta t\|F-F_h\|_{L^\infty}.
\]


\end{lemma}



%-------------------------------------------------------



El método requiere evaluar $f$ en puntos que no coinciden con los nodos de la
malla, por lo que se introducen operadores de proyección e interpolación. Se utiliza
interpolación bilineal, que garantiza un error de segundo orden en el espacio de
fases:

\begin{lemma}[Error de interpolación lineal]

Sea la malla $\{(x_i,\, p_j) : (i,j)\in \mathbb{Z}^2\},$ donde $x_i = i\,\Delta x, p_j = j\,\Delta p.$ Se denota por $I : \ell^\infty(\mathbb{Z}^2) \longrightarrow L^\infty(\mathbb{R}^2), 
\quad 
\Pi : L^\infty(\mathbb{R}^2) \longrightarrow \ell^\infty(\mathbb{Z}^2)$ a los operadores de interpolación lineal y proyección, respectivamente. Sea $u\in W^{2,\infty}(\mathbb{R}^2)$. Entonces


\[
\|(I\circ\Pi)u - u\|_{L^\infty}
\le \frac{1}{8}(\Delta x^2 + \Delta p^2)
\max\left\{
\left\|\frac{\partial^2 u}{\partial x^2}\right\|_{L^\infty},
\left\|\frac{\partial^2 u}{\partial p^2}\right\|_{L^\infty}
\right\}.
\]



\end{lemma}




%-----------------------------------------------------------

\subsection{Actualización de los campos electromagnéticos}

Además de la ecuación de Vlasov, el modelo reducido de Vlasov--Maxwell incluye la
evolución del potencial vector $A(t,x)$ y del campo eléctrico longitudinal $E(t,x)$.
Ambas magnitudes están acopladas a la función de distribución a través de los
momentos macroscópicos, por lo que su actualización debe realizarse de manera
coherente con la evolución de $f$. \\

Para ello se introducen las variables auxiliares


\[
U(t,x)=\partial_t A(t,x), \qquad V(t,x)=\partial_x A(t,x),
\]


que permiten expresar la ecuación de onda para $A$ como un sistema de primer
orden.

El sistema macroscópico queda entonces dado por


\[
\partial_t E = \int_{\mathbb{R}} v(p)\,f(t,x,p)\,dp, \quad \partial_t A = U,
\]




\[
\partial_t U - \partial_x V = -A\,\rho(t,x), \quad
\partial_t V - \partial_x U = 0,
\]


donde


\[
\rho(t,x)=\int_{\mathbb{R}} f(t,x,p)\,dp.
\]



Las dos últimas ecuaciones pueden combinarse para obtener ecuaciones de transporte
para las combinaciones




\[
\partial_t(U+V) - \partial_x(U+V) = -A\rho,
\]




\[
\partial_t(U-V) + \partial_x(U-V) = -A\rho.
\]




Integrando las ecuaciones anteriores a lo largo de las características del transporte se
obtienen las expresiones


\[
U(t,x) = \frac{1}{2}(U+V)(0,x+t) + \frac{1}{2}(U-V)(0,x-t)
\]
\[
- \frac{1}{2}\int_0^t (A\rho)(s,x+t-s)\,ds
- \frac{1}{2}\int_0^t (A\rho)(s,x-t+s)\,ds,
\]




\[
V(t,x) = \frac{1}{2}(U+V)(0,x+t) - \frac{1}{2}(U-V)(0,x-t)
\]
\[
- \frac{1}{2}\int_0^t (A\rho)(s,x+t-s)\,ds
+ \frac{1}{2}\int_0^t (A\rho)(s,x-t+s)\,ds.
\]





En el método semi--Lagrangiano, las variables macroscópicas se actualizan en tiempos
intermedios $t^{n+1/2}$ y $t^{n+3/2}$. El esquema numérico utiliza las siguientes
fórmulas discretas:

\begin{itemize}
\item \textbf{Actualización del potencial vector:}


\[
A^{n+1}_i = A^n_i + \Delta t\,U^{n+1/2}_i.
\]



\item \textbf{Actualización del campo eléctrico:}


\[
E^{n+3/2}_i = E^{n+1/2}_i + \Delta t\sum_j v(p_j)\,f^{n+1}_{ij}\,\Delta p.
\]



\item \textbf{Actualización de $U$ y $V$:}


\[
(U\pm V)^{n+3/2}_i =
I\big((U\pm V)^{n+1/2}\big)(x_i\pm\Delta t)
-\Delta t\,I(A^{n+1})\,I_1(\rho^{n+1})(x_i\pm\Delta t/2),
\]


donde


\[
\rho^{n+1}_i = \sum_j f^{n+1}_{ij}\,\Delta p.
\]


\end{itemize}

Estas expresiones permiten actualizar de forma explícita todos los campos
electromagnéticos a partir de la solución de Vlasov en el nuevo paso temporal.


El siguiente resultado demostrado en \cite{bostan2009convergence} resume el comportamiento de las variables macroscópicas:

\begin{lemma}[Estabilidad de los campos]
Si $A_0\in W^{2,\infty}$, $A_1\in W^{1,\infty}$ y $E_0$ es compatible con la ecuación
de Poisson, entonces las soluciones discretas satisfacen


\[
A^n,\; U^n,\; V^n,\; E^n \in W^{1,\infty},
\]


uniformemente para $n\Delta t\le T$.
\end{lemma}

Este resultado garantiza que el esquema mantiene la regularidad necesaria para que
la aproximación semi--Lagrangiana de las características siga siendo válida en cada
paso temporal.

\medskip



%--------------------------------------------------------------

\subsection{Esquema numérico completo}

El procedimiento completo en cada paso temporal puede resumirse en los siguientes pasos:

\begin{itemize}
\item \textbf{Reconstrucción de la función de distribución.}  
A partir de los valores nodales $f^n_{ij}$, se utiliza el operador de interpolación
bilineal para obtener una aproximación continua $I f^n$ en el espacio de fases.


\item \textbf{Aproximación de las características mediante splitting.}  
El desplazamiento hacia atrás en el tiempo se calcula mediante el esquema de
splitting de segundo orden: combina media advección en espacio, advección
completa en momento y otra media advección en espacio.  
El punto característico aproximado $(\tilde{x},\tilde{p})$ se utiliza para evaluar la solución en el instante anterior.

\item \textbf{Actualización explícita de los campos electromagnéticos.}  
El potencial vector $A$ y el campo eléctrico $E$ se actualizan mediante sus fórmulas
explícitas, mientras que las variables auxiliares $U=\partial_t A$ y
$V=\partial_x A$ se obtienen resolviendo ecuaciones de transporte para
$U\pm V$.  


\item \textbf{Cálculo de los momentos macroscópicos.}  
Una vez obtenida la nueva función de distribución $f^{n+1}$, se calculan los
momentos


\[
\rho^{n+1}(x_i)=\sum_j f^{n+1}_{ij}\,\Delta p, \qquad
j^{n+1}(x_i)=\sum_j v(p_j)f^{n+1}_{ij}\,\Delta p,
\]


que actúan como fuentes en las ecuaciones de Maxwell discretizadas.
\end{itemize}

\medskip

En conjunto, estos cuatro componentes dan lugar a un esquema numérico estable,
capaz de manejar desplazamientos grandes en el espacio de fases y que mantiene un
orden de precisión de segundo orden en el tiempo.






%---------------------------------------------------------------


\subsection{Soporte compacto y estabilidad}

Un aspecto fundamental en la implementación práctica del método semi–Lagrangiano
es el comportamiento del soporte de la función de distribución a lo largo del tiempo.
En el modelo continuo, si la condición inicial $f_0$ tiene soporte compacto en el
espacio de fases, entonces las características permanecen acotadas para todo tiempo
finito, lo que garantiza que la solución $f(t)$ conserva un soporte compacto
uniformemente en $t$. Este hecho es esencial para evitar la aparición de efectos
sin relación física debidos a la truncación del dominio numérico.

En el esquema semi–Lagrangiano la aproximación de
las características mediante splitting y la actualización explícita de los campos
electromagnéticos permiten demostrar que, si la condición inicial discreta
$(f^0_{ij})_{i,j\in\mathbb{Z}}$ tiene soporte compacto, entonces el soporte de
$(f^n_{ij})_{i,j\in\mathbb{Z}}$ permanece uniformemente acotado para todo
$n\Delta t\le T$. En particular, no se generan partículas numéricas fuera del dominio
físico relevante. Este comportamiento se resume en el siguiente resultado demostrado en \cite{bostan2009convergence}:

\begin{lemma}[Conservación del soporte compacto]
Si la condición inicial $f_0$ tiene soporte compacto y los campos iniciales
$(E_0,A_0,A_1)$ son suficientemente regulares, entonces existe un conjunto compacto
$K\subset\mathbb{R}^2$ tal que


\[
\operatorname{supp}(f^n)\subset K \qquad \text{para todo } n\Delta t\le T.
\]


\end{lemma}

La conservación del soporte compacto tiene dos consecuencias directas:

\begin{itemize}
\item \textbf{Estabilidad numérica.}  
El método no genera valores sin correlación física fuera del dominio físico, lo que evita
inestabilidades asociadas a extrapolaciones o a la propagación artificial del
soporte.

\item \textbf{Eficiencia computacional.}  
El dominio numérico puede truncarse de forma segura sin necesidad de imponer
condiciones de frontera artificiales, ya que la solución permanece confinada en
una región finita del espacio de fases.
\end{itemize}

En conjunto, la conservación del soporte compacto y la regularidad uniforme de los
campos electromagnéticos garantizan que el esquema semi–Lagrangiano es estable y
adecuado para el análisis de convergencia que se desarrolla en la siguiente sección.



%---------------------------------------------------------------
\newpage
\section{Estudio de la convergencia}

En esta sección se presentan los resultados principales relativos a la convergencia del
método semi–Lagrangiano aplicado al modelo reducido de Vlasov–Maxwell. El
objetivo es estimar el error cometido por el esquema numérico al aproximar la
solución exacta $(f,E,A)$ del sistema continuo. Para ello se introducen las cantidades:


\[
\Delta f^n_{ij} = f^n_{ij} - f(t^n,x_i,p_j), \qquad
\Delta E^{n+1/2}_i = E^{n+1/2}_i - E(t^{n+1/2},x_i),
\]




\[
\Delta A^n_i = A^n_i - A(t^n,x_i), \qquad
\Delta U^{n+1/2}_i = U^{n+1/2}_i - \partial_t A(t^{n+1/2},x_i),
\]




\[
\Delta V^{n+1/2}_i = V^{n+1/2}_i - \partial_x A(t^{n+1/2},x_i).
\]



A continuación se recogen los resultados fundamentales demostrados en \cite{bostan2009convergence} que permiten controlar estos
errores y demostrar la convergencia del método.\\





\begin{lemma}[Control del error en $f$]
Bajo las hipótesis de los Teoremas 3.1 y 3.2, existe una constante $C>0$ tal que


\[
\|\Delta f^{\,n+1}\|_{l^\infty}
\le \|\Delta f^{\,n}\|_{l^\infty}
+ C\Delta t\big(
\|\Delta E^{\,n+1/2}\|_{l^\infty}
+ \|\Delta A^{\,n}\|_{l^\infty}
+ \|\Delta U^{\,n+1/2}\|_{l^\infty}
+ \|\Delta V^{\,n+1/2}\|_{l^\infty}
\big)
\]




\[
\qquad\qquad
+ C\big(\Delta x^2 + \Delta p^2 + \Delta t^3 + \Delta t\,\Delta x^2\big).
\]


\end{lemma}



\begin{lemma}[Control del error en $A$]
Bajo las hipótesis de los Teoremas 3.1 y 3.2, existe una constante $C>0$ tal que


\[
\|\Delta A^{\,n+1}\|_{l^\infty}
\le \|\Delta A^{\,n}\|_{l^\infty}
+ \Delta t\,\|\Delta U^{\,n+1/2}\|_{l^\infty}
+ C\Delta t^3.
\]


\end{lemma}



\begin{lemma}[Control del error en $E$]
Bajo las hipótesis de los Teoremas 3.1 y 3.2 existe una constante $C>0$ tal que


\[
\|\Delta E^{\,n+3/2}\|_{l^\infty}
\le \|\Delta E^{\,n+1/2}\|_{l^\infty}
+ C\Delta t\,\|\Delta f^{\,n+1}\|_{l^\infty}
+ C\Delta t\,\Delta p^2
+ C\Delta t^3.
\]


\end{lemma}


\begin{lemma}[Control del error en $U\pm V$]
Bajo las hipótesis de los Teoremas 3.1 y 3.2 existe una constante
$C>0$ tal que


\[
\|\Delta(U^{\,n+3/2} \pm V^{\,n+3/2})\|_{l^\infty}
\le
\|\Delta(U^{\,n+1/2} \pm V^{\,n+1/2})\|_{l^\infty}
\]




\[
\qquad\qquad
+ C\Delta t\big(\|\Delta f^{\,n+1}\|_{l^\infty}
+ \|\Delta A^{\,n+1}\|_{l^\infty}\big)
+ C\Delta x^2
+ C\Delta t(\Delta x^2 + \Delta p^2).
\]


\end{lemma}



Para obtener una medida global del error, se introduce la cantidad


\[
e_n := 
\|\Delta f^{\,n}\|_{l^\infty}
+ \|\Delta A^{\,n}\|_{l^\infty}
+ \|\Delta E^{\,n+1/2}\|_
\]
\[
+ \|\Delta(U^{\,n+1/2}+V^{\,n+1/2})\|_{l^\infty}
+ \|\Delta(U^{\,n+1/2}-V^{\,n+1/2})\|_{l^\infty}.
\]



El siguiente teorema proporciona la estimación final de convergencia del método.

\begin{theorem}[Convergencia del esquema]
Bajo las hipótesis de los Teoremas 3.1 y 3.2 existe una constante
$C>0$ tal que, para todo $n$ con $n\Delta t\le T$,


\[
e_n \le
C\big(\Delta t^2 + \Delta x^2 + \Delta p^2 + \Delta t\,\Delta p^2\big)
+ C\,\frac{\Delta x^2 + \Delta p^2}{\Delta t}.
\]


\end{theorem}


Donde el último término $\frac{\Delta x^2 + \Delta p^2}{\Delta t}$ permite realizar una discretización temporal $\Delta t$ muy fina con la única condición de realizar discretizaciones $\Delta x$ y $\Delta p$ de al menos la mitad en orden de magnitud. Lo que permite una precisión temporal muy elevada sin pagar el coste de una discretización espacial aún más fina. Que es exactamente lo contrario a lo que sucede en métodos en diferencias finitas donde la condición usual es $\Delta t \sim \left( \Delta x \right)^2$.




\section{Resultados numéricos}

En esta sección se presentan resultados numéricos que ilustran el
comportamiento del método y confirman las estimaciones teóricas obtenidas.\\

El código empleado en este trabajo se encuentra disponible públicamente en GitHub: \textcolor{blue}{\texttt{https://github.com/Alisama20}}.



\subsection{Reducción del modelo a 1D1V: Vlasov–Poisson y la inestabilidad de los dos haces}



El modelo reducido de Vlasov--Maxwell considerado en este trabajo describe la
dinámica de un plasma en una dimensión espacial, pero incluye un potencial vector
transversal $A(t,x)$ que representa la componente electromagnética transversal del
campo. Este potencial aparece acoplado a la función de distribución a través de la
fuerza


\[
F = -\left(E + A\,\partial_x A\right),
\]


y satisface una ecuación de onda no lineal. Sin embargo, en una dimensión de velocidad el campo magnético desaparece \cite{martin2021campo}.

De esta forma, el sistema reducido de Vlasov--Maxwell colapsa de forma
natural al modelo electrostático de Vlasov--Poisson:


\[
\partial_t f + v\,\partial_x f - E(t,x)\,\partial_v f = 0,
\qquad
\partial_x E(t,x) = \rho(t,x) - \rho_0,
\]


donde $\rho(t,x)=\int_{\mathbb{R}} f(t,x,v)\,dv$ y $\rho_0$ es la densidad media que asegura la neutralidad de carga.
Por lo que este modelo describe la interacción de las partículas únicamente mediante el campo
eléctrico longitudinal.\\


La condición inicial corresponde a dos haces de electrones contrarrotantes, perturbados ligeramente en el espacio:


\[
f_0(x,v) = \frac{1}{\sqrt{2\pi}}\,v^2 e^{-v^2/2}\,\big(1+\alpha\cos(\theta x)\big),
\]


con $\alpha=10^{-2}$ y $\theta=0.5$. La función $v^2 e^{-v^2/2}$ presenta dos máximos
simétricos en velocidad, lo que representa dos poblaciones de partículas moviéndose
en direcciones opuestas. La perturbación espacial de pequeña amplitud desencadena la conocida \textit{inestabilidad de dos haces}.

El campo eléctrico se obtiene resolviendo la ecuación de Poisson en el espacio de
Fourier, y la ecuación de Vlasov se integra mediante el esquema semi--Lagrangiano
con splitting de Strang descito (medio paso en velocidad, paso completo en espacio y otro medio paso en velocidad)



En la Figura~\ref{doshaces} se representan estados de la función de
distribución $f(t,x,v)$ en el plano $(x,v)$ para los tiempos $t=0$, $t=15$, $t=20$ y
$t=30$. La evolución observada es característica de la inestabilidad de dos haces:

\begin{itemize}
    \item \textbf{En $t=0$} se aprecian los dos máximos en velocidad,
    correspondientes a los dos haces iniciales, con una nula diferencia espacial.

    \item \textbf{En $t\approx 15$} comienza la fase lineal de la inestabilidad: la
    perturbación espacial crece exponencialmente y el campo eléctrico se amplifica.
    La distribución empieza a deformarse en el espacio de fases.

    \item \textbf{En $t\approx 20$} aparecen
    estructuras finas y filamentos en el plano $(x,v)$, resultado de la mezcla de los
    haces y del crecimiento del modo inestable.

    \item \textbf{En $t\approx 30$} se observan vórtices electrostáticos y estructuras
    cerradas. La distribución está fuertemente mezclada
    y la inestabilidad ha saturado.
\end{itemize}

\newpage

\begin{figure}[H]
    \centering
    \includegraphics[width=0.9\linewidth]{doshaces.png}
    \caption{El dominio espacial es periódico, $x\in[0,L_x]$ con $L_x=4\pi$, discretizado con
$N_x=200$ puntos. El dominio en velocidades es $v\in[-v_{\max},v_{\max}]$ con
$v_{\max}=6$ y $N_v=200$ puntos. El paso de tiempo se fija en $\Delta t = 0.05$. }
    \label{doshaces}
\end{figure}



%---------------------------------------------------------------




\subsection{Oscilación de Langmuir relativista en el modelo QR 1D2V}

El modelo se amplía ahora para considerar dos componentes de momento
$(p_x,p_y)$ y un potencial vector transversal $A_y(t,x)$ que genera un
campo electromagnético transversal. La función de distribución $f(t,x,p_x,p_y)$
satisface la ecuación de Vlasov cuasi-relativista:

\[
\partial_t f + \frac{p_x}{\gamma}\,\partial_x f 
+ F_x\,\partial_{p_x} f + F_y\,\partial_{p_y} f = 0,
\]

donde $\gamma = \sqrt{1 + p_x^2 + p_y^2}$ y la fuerza de Lorentz se expresa como

\[
F_x = E_x + \frac{p_y}{\gamma}\,B_z, \qquad
F_y = -\frac{p_x}{\gamma}\,B_z,
\]

con $E_x$ el campo eléctrico longitudinal obtenido de Poisson y $B_z = \partial_x A_y$ el campo magnético transversal.  
El potencial vector $A_y$ evoluciona según la ecuación de onda no lineal acoplada a la corriente transversal

\[
\partial_{tt} A_y - \partial_{xx} A_y = - J_y, 
\quad J_y = \int \frac{p_y}{\gamma} f\,dp_x dp_y.
\]

\newpage

Para integrar numéricamente el sistema, se ha seguido la misma estrategia semi-Lagrangiana que en 1D1V, con splitting de Strang:

\begin{enumerate}
    \item Medio paso en velocidades $(p_x,p_y)$ usando interpolación cúbica (Catmull--Rom) a lo largo de la fuerza de Lorentz.
    \item Paso completo en espacio $x$ usando interpolación cúbica periódica.
    \item Medio paso final en velocidades.
\end{enumerate}




La simulación se inicia a partir de una distribución de equilibrio de tipo
Maxwelliano en el espacio de momentos, perturbada mediante una modulación
sinusoidal de pequeña amplitud en la variable espacial. En concreto, la
distribución inicial se define como
\begin{equation}
f_0(x,p_x,p_y)
= \frac{1}{2\pi T}\,
\exp\!\left(-\frac{p_x^2 + p_y^2}{2T}\right)
\left[1 + \alpha \cos(k_0 x)\right],
\label{eq:CI_f0}
\end{equation}
donde $T=0.1$ es la temperatura, $\alpha = 0.01$ la amplitud de
la perturbación y $k_0 = 0.5$ el número de onda inicial. El dominio espacial es
periódico, $x\in[0,L_x]$ con $L_x = 4\pi$.

A partir de $f_0$ se calcula la densidad inicial


\[
\rho(x,0) = \int_{\mathbb{R}^2} f_0(x,p_x,p_y)\,dp_x\,dp_y,
\]


y el campo eléctrico longitudinal inicial $E_x(x,0)$ se obtiene resolviendo la
ecuación de Poisson.

El potencial vector transversal se inicializa como


\[
A_y(x,0) = 0, \qquad A_y(x,-\Delta t) = 0,
\]


de modo que el campo magnético inicial es nulo,


\[
B_z(x,0) = \partial_x A_y(x,0) = 0.
\]



Finalmente, la distribución inicial se normaliza para que la masa total


\[
M_0 = \int f_0(x,p_x,p_y)\,dx\,dp_x\,dp_y
\]


sea exactamente unitaria en la discretización numérica.


\newpage


\begin{figure}[H]
    \centering
    \begin{subfigure}[b]{0.48\textwidth}
        \centering
        \includegraphics[width=\textwidth]{t10.png}
        \caption{$t = 10.0$}
        \label{fig:t10}
    \end{subfigure}
    \hfill
    \begin{subfigure}[b]{0.48\textwidth}
        \centering
        \includegraphics[width=\textwidth]{t20.png}
        \caption{$t = 20.0$}
        \label{fig:t20}
    \end{subfigure}
    
    \vspace{0.5cm}
    
    \begin{subfigure}[b]{0.48\textwidth}
        \centering
        \includegraphics[width=\textwidth]{t60.png}
        \caption{$t = 60.0$}
        \label{fig:t60}
    \end{subfigure}
    \hfill
    \begin{subfigure}[b]{0.48\textwidth}
        \centering
        \includegraphics[width=\textwidth]{t100.png}
        \caption{$t = 100.0$}
        \label{fig:t100}
    \end{subfigure}
    
    \caption{Densidad de probabilidad $f(t,x,p_x)$ en $p_x = 0$ a distintos tiempos. El sistema se ha simulado resolviendo el modelo reducido de Vlasov--Maxwell en 1D2V mediante un esquema semi-Lagrangiano con splitting de Strang. El dominio espacial es periódico, $x\in[0,L_x]$ con $L_x=4\pi$, discretizado con $N_x=200$ puntos. El dominio en velocidades es $v\in[-v_{\max},v_{\max}]$ con $v_{\max}=0.6 c$ y $N_v=200$ puntos. El paso de tiempo es $\Delta t = 0.05$.} 
    \label{fig:densidad_px_py0}
\end{figure}








En la Figura \ref{fig:densidad_px_py0} se observa como la presencia del campo magnético transversal $B_z$ modifica la dinámica de la distribución, incluyendo acoplamientos en $p_y$ y $p_x$ que inducen una rotación en las partículas del sistema. Se observa que $f$ manifiesta una oscilación, muy similar al caso 1D1V.



%--------------------------------------------------------------


\newpage

\section{Conclusión}

La dinámica cinética de plasmas constituye uno de los problemas más desafiantes
en física matemática y computacional. Aunque las ecuaciones de Vlasov--Maxwell
describen con exactitud la evolución de un plasma sin colisiones, su complejidad
hace necesario recurrir a métodos numéricos capaces de capturar la estructura
fina del espacio de fases sin introducir artefactos que distorsionen la dinámica.
En este trabajo se ha desarrollado y analizado un esquema semi--Lagrangiano de
alta precisión para las versiones reducidas 1D1V y 1D2V del modelo, incluyendo
su formulación cuasi--relativista.

El análisis teórico ha permitido establecer un marco claro para el estudio del
error asociado al método. A partir de estimaciones locales sobre la función de
distribución, el potencial vectorial y el campo eléctrico, se ha construido una
medida global del error cuya evolución queda controlada por términos de orden
$\Delta t$, $\Delta x^2$ y $\Delta p^2$. El teorema de convergencia resultante garantiza
la estabilidad del esquema bajo hipótesis de regularidad razonables.

En la parte numérica se han considerado dos escenarios representativos. El
primero corresponde al modelo electrostático de Vlasov--Poisson, donde se ha
reproducido la inestabilidad de dos haces. Las simulaciones muestran el
crecimiento exponencial inicial, la transición al régimen no lineal y la
formación de estructuras tipo vórtices. Este caso demuestra la capacidad del método para resolver
estructuras finas en el espacio de fases.

El segundo escenario aborda la oscilación de Langmuir relativista en el modelo
cuasi--relativista 1D2V. Los resultados obtenidos muestran oscilaciones debidas a la presencia de campo magnético, lo que hace notar la consistencia del transporte en dos
dimensiones de momento. La evolución de los cortes $f(t,x,p_x,p_y=0)$
reproduce la dinámica esperada de una onda de Langmuir relativista, validando el
modelo en un régimen más exigente.\\

El esquema semi--Lagrangiano desarrollado se presenta como una
herramienta eficaz para la simulación de plasmas sin colisiones en geometrías
reducidas. 







%----------------------------------------------------------------

\newpage
% Referencias %%%%%%%%%%%%%%%%%%%%%%%%%%%%%%%%%%%%%%%%%%%%%%%%%%%%%%%%%%%%%%%%%
%\newpage

\addcontentsline{toc}{section}{Referencias} % Elige según idioma
%\addcontentsline{toc}{section}{References} % Elige según idioma



\bibliographystyle{plain}
\bibliography{bibliografia}



\end{document}

